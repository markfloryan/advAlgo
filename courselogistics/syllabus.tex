\documentclass[12pt]{article}
\usepackage[margin=1in]{geometry}                % See geometry.pdf to learn the layout options. There are lots.
\geometry{letterpaper}                   % ... or a4paper or a5paper or ... 
\usepackage[parfill]{parskip}    % Activate to begin paragraphs with an empty line rather than an indent
\usepackage{graphicx}
\usepackage{diagbox}
\usepackage{amsthm}
\usepackage{amsmath}
\usepackage{amssymb}
\usepackage{algorithm}
\usepackage[noend]{algpseudocode}
\usepackage{mdframed}
\usepackage{epstopdf}
\usepackage[font=footnotesize]{caption}
\usepackage{subcaption}
\usepackage{cite}
\usepackage{color}
\usepackage[dvipsnames]{xcolor}
\usepackage{bbding}
\usepackage[hidelinks]{hyperref}
\usepackage{verbatim}
\usepackage{comment}
\graphicspath{{figures/}{pictures/}{images/}{./}} % where to search for the images
\DeclareGraphicsExtensions{.pdf,.png,.jpg,.jpeg,.eps} % for pdflatex we expect .pdf, .png, or .jpg files
\DeclareGraphicsRule{.tif}{png}{.png}{`convert #1 `dirname #1`/`basename #1 .tif`.png}

\newcommand{\note}[1]{{\color{blue} \textit{note: #1}}}
\newcommand{\done}{{\color{green} \CheckmarkBold}}
\newcommand{\timeline}[1]{{\color{red} -- #1}}
\begin{document}

%\textbf{\Large UNDER CONSTRUCTION!!!}\\
\textbf{\Large Advanced Algorithms and Imp.} \hfill \textbf{\Large Spring 2026}

\vskip 0.5in 

\makebox[\textwidth][c]{
\begin{tabular}{p{2.6in}p{2.6in}}
    \textbf{TuTh 11:00 am--12:15 pm } \\
    \textbf{Instructor:} Mark Floryan & \\
    Email: \url{mfloryan@cs.virginia.edu} & \\
    Office: Rice Hall 203 & \\
    Office Hours: TBD. See course website & \\
%   &  \hskip 0.2in  Mo/We 1:30--2:30 pm; and,\\
%   &  \hskip 0.2in Tu noon--12:45 pm; and,\\
%   &  \hskip 0.2in  Fr 11:00--noon\\
%     \hskip 0.2in Tu 3:00p-4:00p & \hskip 0.2in Tu/Th 10:30-11:30a,\\
%     \hskip 0.2in  & \hskip 0.2in Th 1:00-2:00p  %\\
    %Regrades: tbd & Regrades: tbd 
    %\hskip 0.2in Th 11:00a-12:30p (CS 4102) & \hskip 0.2in W 4:00p-6:00p \\
    %\hskip 0.2in {\color{darkgray} W 2:00p-3:30p (CS 2110)} & \hskip 0.2in \\
    %Regrades: Tu 4:00p-5:00p & Regrades: Tu 4:00p-5:00p
\end{tabular}}

\vskip 0.1in
\textbf{Teaching Assistants:} See course website.  Office hours will be done in-person.

\vskip 0.1in
\textbf{Course website:} {\tt https://markfloryan.github.io/advAlgo/} (We will also use Collab.)

\textbf{Prerequisites:} CS 3100 (or equivalent) with grades of C- or higher, and math knowledge from APMA 1090 or MATH 1210 or MATH 1310. (Prerequisites are important to this course and will be enforced!)

\section*{Overview}

\textbf{Course Description:} Builds upon the foundational material in data structures and algorithms. Introduces advanced applications of data structures such as fenwick trees, quad trees, and Van Emde Boas trees. Introduces advances algorithmic strategies such as Linear Programming, Computational Geometry algorithms, and approximation algorithms.

\textbf{Availability:} It is important to us to be available to our students, and to address their concerns.  If you cannot meet with either of us during our office hours, e-mail us and we will find the time to meet. That being said, like everybody else we are quite busy, so it may take a day or more to find a time to meet. And if you have any comments on the course---what is working, what is not working, what can be done better, etc.---we are very interested in hearing about them.  Please send Prof.\ Floryan or one of the TAs an e-mail or post privately on Piazza to the instructors.  When sending email, include ``Advanced Algo'' in the subject line. If your question could be answered by either professor or even a TA, a post on Piazza to "instructors" may get a faster response.


\textbf{Course Objectives:} Students who complete the course will:
\begin{itemize}
    \item Solve challenging optimizations by modeling problems as linear programs and using / understanding linear programming algorithms. 
    \item Take advantage of representation structures of data to more efficiently store and retrieve data.
    \item Apply analysis concepts to determine the efficacy of approximation algorithms where the optimal solution is unlikely to be found.
    \item Gain an understanding of common algorithmic ideas when working with geometric problems including convex hulls, and large geometric data sets.
    \item Be exposed to numerous small but useful algorithms that can be used as common sub-routines to solve larger problems.
\end{itemize}

\textbf{Textbook:} \textit{Introduction to Algorithms, Third Edition} by Cormen, et. al. (ISBN 0262033844).\\
UVA Library makes a digital version of our textbook available online at\\\url{https://search.lib.virginia.edu/catalog/u6757775}\\
%\textbf{Note:} How much we ask you to read from this textbook may vary between instructors. Your instructor will say more about this in class.

\textbf{Additional Resources:} We will also be using the CP-Algorithms website, which is a curated library of advanced algorithms (found at \url{https://cp-algorithms.com})

\section*{Class Delivery}

Lectures and quizzes will be given in-person.  (If the university changes its policy due to changing circumstances, this may change. We will follow university guidance in such matters.) We will do our best to make recordings of lectures available on the Collab site.

We will follow the university's guidance on dealing with Covid, including wearing masks while indoors. See the course website's page on the course's policies on dealing with Covid-19.

%Each lecture in this course will be split into two halves, a asynchronous recorded portion and a live in-person portion. The total combined time for these lecture portions will be equal (or very close) to the normal lecture time (1 hour, 15 minutes).
%\begin{itemize}
%\item The asynchronous recorded lecture  is intended to introduce the basic core material for each lecture. Students can watch this before class, or during the first half of class, before the in-person portion begins.
%\item The live in-person portion of class will take place on Zoom, and will focus on extra examples, more complicated proofs, and other such things that require or benefit from an in-person explanation.
%\end{itemize}
%Because of this split, the first 35 minutes of lecture will be reserved for students who wish to watch the recorded lecture during reserved class time. For example, if class is scheduled from 11--12:15, there will not be class from 11--11:35 so students may watch the recording (or students may watch it ahead of time, up to you). In-person class begins at 11:35 and will proceed until 12:15.


\section*{Coursework and Grading}

The course is divided into 7 overall {\bf modules / topics}:
\begin{itemize}
    \item Fenwick Trees
    \item Segment Trees
    \item Linear Programming
    \item Computational Geometry
    \item Van Emde Boas Trees
    \item Approximation Algorithms
    \item Little Algorithms (Final Project)
\end{itemize}

Most modules are 2-5 lectures worth of content. The schedule is shown on the course website. 

\textbf{Quizzes:}  There will be two short quizzes throughout the semester, which will be taken on two days throughout the term. Each quiz is a short assessment of your knowledge of 1-3 modules / topics. Quizzes are meant to ensure you clearly demonstrate competence regarding the individual topics for those modules.

Please consult the course website for the specific dates on which quizzes will occur:\\

\textbf{Quiz Makeup Policy:} Quiz makeups, in general, will not be allowed for any reason. Should you not be able to attend a quiz day due to a legitimate reason (illness, family emergency, religious observance, etc.) then a makeup quiz will be provided during the scheduled final exam time. This makeup quiz will be a cumulative quiz that covers all topics for the course and will be slightly (though not significantly) longer than a normal quiz.

\textbf{Homeworks:} All homework assignments will be advanced ``programming challenges.'' Programming assignments can be implemented in Python, Java or C++. Homeworks will be submitted on Gradescope (see Collab for link to Gradescope site) and auto-graded upon submission.

\textbf{Grading:} This course will use a standard weighted-average grading system. The grade breakdown is shown below:

\begin{itemize}
    \item \textbf{Homework (40\%)}: There will be six programming challenges which account in total for 40 percent of the final grade.
    \item \textbf{Quizzes (30\%)}: There will be two in-class quizzes which account for 30 percent of the final grade.
    \item \textbf{Attendance / Participation (12\%)}: Attendance and participation are required in this course. Every lecture, a random subset of students will be chosen by a computer program for attendance. If your name is chosen and you are not present, you will lose 2 of these 12 percentage points.
    \item \textbf{Final Project (18\%)}: There will be a final project in which groups of students study, implement, prepare, and present one advanced algorithm of their choosing. Details will be available on the course website.
\end{itemize}


\textbf{Submission System:} All work will be submitted via GradeScope. Details will be explained later in the course. 

\textbf{Homework Late Policy:} For programming challenges, due dates will be set on the course website. Students may submit the assignment up to 5 days late with no penalty. There will be NO extensions provided beyond this 5 day window for any reason. The 5 day window is the grace period students may use and are expected to do so wisely / at their own risk.

\textbf{Regrades:} Regrades will be provided for quizzes. An announcement will be made regarding these details after each quiz is graded.

\section*{Collaboration Policy}

\textbf{Quizzes:} Quizzes are always individual assignments; collaboration with others is not allowed. Any solutions that share similar text or code will be considered in breach of this policy.

\textbf{Homeworks:} Programming challenges must be done individually (i.e., each student will submit their own individual code to Gradescope). There will be instances when we are working on homework assignments together in class. During these special class periods, collaboration with other students is fully allowed as long as code is not being directly copied from one student to another. In general, you can discuss the high-level ideas / solutions to problems openly with others, but no code can be viewed / shared across students.

We encourage you \textbf{not} to seek published or online solutions for any assignment, since this is not a good way to learn. Studying code online is permitted, but only for getting ideas about how to address a programming solution. 
Copying or reusing code from an online source violates the honor pledge for that homeowrk. 
You must cite sources of any online code you use in this way in a comment in your source file(s)

Any submission which is discovered to be similar to a published solution or one found online will be considered in breach of this policy.

Note that it is a violation of this policy to submit a problem solution that you are unable to explain orally to a member of the course staff, and we reserve the right to spot-check for this requirement and to use tools like {\tt moss} etc. to detect shared code.

\textbf{Penalty:} Assignments or quizzes where violations of this policy occur will receive a \textbf{zero} grade for that assignment + one additional assignment of equivalent rank (homework, quiz, etc.) will be set to a zero as well. A second infractions will result in a \textbf{failing (F)} grade in the course.  Any infractions will also be submitted to the Honor Committee if deemed appropriate.

\section*{Additional Information}

%\textbf{Inclement weather, power outages, etc.:} Online classes may still be affected by inclement weather or power outages.  Our class will proceed as normal even if the university cancels in-person classes. We will record every ``live'' session of class, so if a student loses power and cannot attend they can view the recording later.  If neither of the professors can host a live session because of power outages, we'll use our phones to announce that using Collab.

\textbf{Special Circumstances:} The University of Virginia strives to provide accessibility to all students. If you require an accommodation to fully access this course, please contact the Student Disability Access Center (SDAC) at (434) 243-5180 or \url{sdac@virginia.edu}. If you are unsure if you require an accommodation, or to learn more about their services, you may contact the SDAC at the number above or by visiting their website \url{http://studenthealth.virginia.edu/sdac}.

For this course, we ask that students with special circumstances let us know as soon as possible, preferrably during the \textbf{first week of class}.

\textbf{Religious Accommodations:} It is the University's long-standing policy and practice to reasonably accommodate students so that they do not experience an adverse academic consequence when sincerely held religious beliefs or observances conflict with academic requirements.  Students who wish to request academic accommodation for a religious observance should submit their request in writing to Prof. Floryan as far in advance as possible. If you have questions or concerns about academic accommodations for religious observance or religious beliefs, visit 

\begin{center} 
    \url{https://eocr.virginia.edu/accommodations-religious-observance}
\end{center}

or contact the University's Office for Equal Opportunity and Civil Rights (EOCR) at \url{UVAEOCR@virginia.edu} or 434-924-3200.  Accommodations do not relieve you of the responsibility for completion of any part of the coursework missed as the result of a religious observance.

\textbf{Safe Environment:} The University of Virginia is dedicated to providing a safe and equitable learning environment for all students. To that end, it is vital that you know two values that we and the University hold as critically important:
 
\begin{enumerate}
    \item Power-based personal violence will not be tolerated. 
    \item Everyone has a responsibility to do their part to maintain a safe community on Grounds.
\end{enumerate}

If you or someone you know has been affected by power-based personal violence, more information can be found on the UVA Sexual Violence website that describes reporting options and resources available -- \url{www.virginia.edu/sexualviolence}. 
   
As your professor and as a person, know that we each care about you and your well-being and stand ready to provide support and resources as we can. As a faculty member, we are responsible employees, which means that we are required by University policy and federal law to report what you tell us to the University's Title IX Coordinator. The Title IX Coordinator's job is to ensure that the reporting student receives the resources and support that they need, while also reviewing the information presented to determine whether further action is necessary to ensure survivor safety and the safety of the University community. If you would rather keep this information confidential, there are Confidential Employees you can talk to on Grounds (See \url{http://www.virginia.edu/justreportit/confidential\_resources.pdf}). The worst possible situation would be for you or your friend to remain silent when there are so many here willing and able to help.

\textbf{Well-being:} If you are feeling overwhelmed, stressed, or isolated, there are many individuals here who are ready and wanting to help. The Student Health Center offers Counseling and Psychological Services (CAPS) for all UVA students. Call 434-243-5150 (or 434-972-7004 for after hours and weekend crisis assistance) to get started and schedule an appointment. If you prefer to speak anonymously and confidentially over the phone, Madison House provides a HELP Line at any hour of any day: 434-295-8255.

\textbf{Syllabus Note:} This syllabus is to be considered a reference document that may be adjusted throughout the course of the semester to address necessary changes. This syllabus can be changed at any time without notification; it is up to the student to monitor the website for news of any changes. Final authority on any decision in this course rests with the professor, not with this document.

\textbf{Research:}
Your class work might be used for research purposes. For example, we may use anonymized scores from student assignments to compare to other student performance data. Any student who wishes to opt out can contact the instructor or TA to do so after final grades have been issued. This has no impact on your grade in any manner. 



\end{document}  
